\section{Structures de données d’arbres syntaxiques abstraits}

Un arbre syntaxique abstrait, d'un point de vue programmation, n'est qu'un
arbre binaire dont les différents membres stockent des informations sur eux-
mêmes. Pour plus de détails sur l'implémentation, le code Java est fourni en
annexe du rapport, mais n'est pas transcrit textuellement pour garder le
rapport concis.

\subsection{ElemAST (abstraite)}

Tout élément d'un arbre fait partie de cette classe. Cela permet d'agglomérer
et d'imposer, si nécessaire, certaines fonctions telles que \verb|toString|,
\verb|asPostFix|, etc. De plus, un nœud ne sait pas si ses branches
contiendront d'autres nœuds ou des feuilles. Cette classe permet de faire
abstraction de cette distinction.

\subsection{NoeudAST}

Un nœud contient les informations suivantes :

\begin{itemize}
  \item \verb|type| : (toujours un opérateur dans notre cas, mais pourrait être
        un appel à une fonction dans une grammaire différente)
  \item \verb|lexeme| : permet d'évaluer l'expression. Omettre un lexème comme
        \verb|*| rendrait impossible de connaître l'opération désirée lors de
        l'évaluation.
  \item \verb|left| : contient le terme gauche du nœud (feuille ou un autre nœud)
  \item \verb|right| : contient le terme droit du nœud (comme \verb|left|)
\end{itemize}

\subsection{FeuilleAST}

Une feuille contient les informations suivantes :

\begin{itemize}
  \item \verb|type| : litéral ou identificateur dans la grammaire actuelle, mais
        pourrait contenir autre chose comme booléen, flottant, etc.
  \item \verb|lexeme| : tout comme dans le nœud, il serait impossible
        d'évaluer l'arbre par la suite s'il n'était pas présent.
\end{itemize}
