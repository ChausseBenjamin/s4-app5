\section{Conception de l'analyseur syntaxique par la méthode descendante}

\subsection{Principes de fonctionnement}

\subsubsection{Analyseur descendant}

Un analyseur syntaxique descendant construit l'\textit{AST} à partir d'un
symbole de départ. Il dérive l'entrée de gauche à droite ou de droite à
gauche en appliquant des règles de grammaire. Ces règles permettent de choisir
le bon prochain symbole ou la bonne combinaison de symboles. Une méthode peut
implémenter des \textit{lookahead tokens} pour déterminer les symboles
suivants et choisir la règle de production adéquate. L'analyseur essaie
plusieurs règles de production jusqu'à ce que la chaîne d'entrée soit analysée
ou qu'une erreur soit détectée.

\subsection{Analyseur LL}

Le premier L de la méthode LL indique que l'entrée est lue de gauche à droite.
Le second L signifie que les dérivations sont effectuées à gauche. Dans
l'énoncé de la problématique, on indique d'utiliser LL(1). Le nombre entre
parenthèses indique la quantité de \textit{lookahead tokens} disponibles.
Dans ce cas, un seul. Cela signifie que les fonctions d'analyse ne peuvent lire
que le caractère actuel et un seul suivant.

La méthode LL est prédictive. Elle anticipe la règle à utiliser et utilise des
appels récursifs aux règles de production.

L'analyseur commence à gauche de l'expression, avec le symbole de départ. Il
regarde le caractère suivant, et s’il peut répondre à une règle avec ce
caractère, il crée un nœud d'arbre avec deux symboles ; sinon, il passe au
suivant. Plus loin dans ce document, la grammaire utilisée démontre ce
principe.

\subsection{Catégorie LL la plus utilisée en pratique}

La méthode la plus utilisée en pratique est \verb|LL(1)|. Cela s'explique par
sa simplicité d'implémentation, souvent via une descente récursive comme
celle de l'APP. Elle est suffisante pour de nombreux langages de
programmation à la syntaxe robuste.

Des méthodes beaucoup plus complexes existent pour traiter des syntaxes
ambiguës ou des grammaires complexes.

\subsection{Type de grammaires applicables}

Les grammaires pour la méthode \verb|LL(1)| ne doivent pas contenir
d’ambiguïtés. Étant donné que la lecture est de gauche à droite, il ne doit
pas y avoir de récursivité à gauche, seulement à droite. De plus, les ensembles
de premiers et de suivants doivent être disjoints. Les grammaires dites
``LL(1)'' sont utilisées pour cette méthode. Les règles de production doivent
aussi être associatives à droite, car il n’est pas possible de vérifier
l'ensemble de la règle avec un seul \textit{lookahead token} dans le cas
d’une associativité à gauche.

\subsection{Définition formelle de la syntaxe d’expressions arithmétiques}

Cette section présente la grammaire utilisée pour résoudre des équations
arithmétiques sans nombres à virgule, sans nombres négatifs, avec parenthèses.
La priorité des opérations stipule que la division et la multiplication ont
la même priorité, supérieure à celle de l'addition et la soustraction.

Comme la méthode doit construire à gauche, les règles de production récursives
sont réécrites pour une récursivité à droite. La construction descendante
nécessite une définition allant du moins prioritaire au plus prioritaire.

\begin{align}
E &\rightarrow \textrm{T | [ +E | -E ]} \\
T &\rightarrow \textrm{P | [ *T | /T ]} \\
P &\rightarrow \textrm{( E )} \\
P &\rightarrow \textrm{délimiteur | littéral}
\end{align}

\subsection{Exemples de séquences de dérivations}

L'exemple suivant montre une séquence de dérivations d'une expression
arithmétique à l'aide de la grammaire définie ci-dessus :

\begin{equation}
1+2*(3/5)-6
\end{equation}

\begin{center}
\begin{minipage}{0.8\textwidth}
  \begin{enumerate}
    \setlength{\itemsep}{0pt}
    \setlength{\parskip}{0pt}
    \begin{multicols}{2}
      \item $E$
      \item $T+E$
      \item $P+E$
      \item $l+E$
      \item $l+T$
      \item $l+P*T$
      \item $l+l*T$
      \item $l+l*P$
      \item $l+l*(E)$
      \item $l+l*(T)$
      \item $l+l*(P/T)$
      \item $l+l*(l/T)$
      \item $l+l*(l/P)$
      \item $l+l*(l/l)$
      \item $l+l*(l/lE$
      \item $l+l*(l/lT-E$
      \item $l+l*(l/lP-E$
      \item $l+l*(l/l)-E$
      \item $l+l*(l/l)-T$
      \item $l+l*(l/l)-P$
      \item $l+l*(l/l)-l$
    \end{multicols}
  \end{enumerate}
\end{minipage}
\end{center}

Lorsqu'il n'est plus possible d'appliquer les règles de production de la
grammaire, et qu'il reste des unités lexicales, l'analyseur recommence à
partir du caractère actuel. C’est pour cela qu’un symbole $E$ apparaît à la
position de la parenthèse fermante. La fonction $P$ s’est assurée qu’une
parenthèse fermante suivait l’appel de $E$, puis l'analyse est remontée et a
repris à partir de ce point.
