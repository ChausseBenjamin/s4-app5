\section{Validation et tests}

\begin{center} %% ANALYSE LEXICALE
  \footnotesize
  \begin{longtable}{cp{4cm}p{4cm}p{5cm}}
    % Headers & Footers {{{
    \caption{Tests d'analyse lexicale} \label{tab:lex}
    \\
    \toprule
    \multicolumn{3}{l}{Objectif Ciblé} &
    Test des nouvelles opérations
    \\
    \midrule
    {\scriptsize \# Test}      &
    \bfseries Entrée           &
    \bfseries Résultat Attendu &
    \bfseries Résultat Obtenu  \\
    \midrule
    \endfirsthead

    \multicolumn{4}{c}%
    {{\itshape \tablename\ \thetable{} -- Continué de la page précédente\ldots}}
    \\
    \midrule
    {\scriptsize \# Test}      &
    \bfseries Entrée           &
    \bfseries Résultat Attendu &
    \bfseries Résultat Obtenu  \\
    \midrule
    \endhead

    \midrule \multicolumn{4}{r}{{Continué à la prochaine page}} \\
    \midrule
    \endfoot

    \bottomrule
    \endlastfoot
    % }}}
    % Tests {{{
    \tid  &
    \verb|(U_x + V_y ) * W_z / 35| &
    Les unités sont présentées sur une ligne contenant le type et le contenu
    du lexème. Le résultat attendu est :
    \begin{itemize}
      \item Délimiteur:     \verb|(|
      \item Identificateur: \verb|U_x|
      \item Opérateur:      \verb|+|
      \item Identificateur: \verb|V_y|
      \item Délimiteur:     \verb|)|
      \item Opérateur:      \verb|*|
      \item Identificateur: \verb|W_z|
      \item Opérateur:      \verb|/|
      \item Littéral:       \verb|35|
    \end{itemize} &
    Le programme répond à ce critère avec une présentation légèrement
    différente. Toutefois, les données sont présentes et dans le bon ordre. \\

    \tid  &
    \verb|(U_x + V_y ) * W__z / 35| &
    Même unités que le premier test. Toutefois, une erreur est donnée,
    indiquant que des underscores consécutifs ont été trouvés. Les unités
    affichées s'arrêtent à l’opérateur de multiplication. &
    Le programme a donné une erreur et a retourné les unités lexicales jusqu’à
    l’opérateur de multiplication. \\

    \tid  &
    \verb|&| &
    Une erreur de caractère invalide est attendue. &
    Le programme identifie le caractère invalide. \\

    \tid  &
    \verb|aA_z| &
    Une erreur est attendue indiquant qu’un identificateur ne peut pas
    commencer par une lettre minuscule. &
    Une erreur d’identificateur est donnée indiquant que le nom ne peut pas
    commencer par une lettre minuscule. \\

    \tid  &
    \verb|Aa_| &
    Une erreur est attendue, indiquant qu’un identificateur ne peut pas se
    terminer par un underscore. &
    Une erreur de fin d’identificateur a été donnée, indiquant qu’il ne peut
    pas se terminer par un underscore. \\
    % }}}
  \end{longtable}
\end{center}

\begin{center} %% ANALYSE SYNTAXIQUE
  \footnotesize
  \begin{longtable}{cp{4cm}p{4cm}p{5cm}}
    % Headers & Footers {{{
    \caption{Tests d'analyse syntaxique} \label{tab:syntax}
    \\
    \toprule
    \multicolumn{3}{l}{Objectif Ciblé} &
    Test des nouvelles opérations
    \\
    \midrule
    {\scriptsize \# Test}      &
    \bfseries Entrée           &
    \bfseries Résultat Attendu &
    \bfseries Résultat Obtenu  \\
    \midrule
    \endfirsthead

    \multicolumn{4}{c}%
    {{\itshape \tablename\ \thetable{} -- Continué de la page précédente\ldots}}
    \\
    \midrule
    {\scriptsize \# Test}      &
    \bfseries Entrée           &
    \bfseries Résultat Attendu &
    \bfseries Résultat Obtenu  \\
    \midrule
    \endhead

    \midrule \multicolumn{4}{r}{{Continué à la prochaine page}} \\
    \midrule
    \endfoot

    \bottomrule
    \endlastfoot
    % }}}
    % Tests {{{
    \tidlabel{tid:start-op} &
    \verb| + 2 - 3| &
    Une expression ne peut pas commencer par un opérateur. Une erreur devrait
    survenir. &
    Voir l'erreur \ref{err:start-op} \\

    \tidlabel{tid:end-op}  &
    \verb| 2 - 3 *| &
    Un opérateur est utilisé sans opérande à droite. Une erreur devrait
    survenir. &
    Voir l'erreur \ref{err:end-op} \\

    \tidlabel{tid:double-op}  &
    \verb|1 +/ 2 - 3| &
    Deux opérateurs consécutifs sont invalides. Une erreur devrait survenir. &
    Voir l'erreur \ref{err:double-op} \\

    \tidlabel{tid:missing-close}  &
    \verb|(1 + 2 - 3| &
    Une parenthèse ouverte n'a pas été fermée. Une erreur devrait survenir. &
    Voir l'erreur \ref{err:missing-close} \\

    \tidlabel{tid:missing-open}  &
    \verb|1 + 2) - 3| &
    Une parenthèse fermante apparaît alors qu'aucune n’a été ouverte. Une
    erreur devrait survenir. &
    Voir l'erreur \ref{err:missing-open} \\

    \tidlabel{tid:empty-paren}  &
    \verb|1 + 2 - () / 3| &
    Les parenthèses doivent contenir une expression valide. Une erreur devrait
    survenir. &
    Voir l'erreur \ref{err:empty-paren} \\
    % }}}
  \end{longtable}
\end{center}

Des tests syntaxiques avec une syntaxe valide n'ont pas été effectués puisque
tous les tests de construction d'AST les emploient (pré-requis pour ce type de
test).

\begin{center} %% CONSTRUCTION AST
  \footnotesize
  \begin{longtable}{cp{4cm}p{4cm}p{5cm}}
    % Headers & Footers {{{
    \caption{Tests de construction d'arbres syntaxiques abstraits}
    \label{tab:ast}
    \\
    \toprule
    \multicolumn{3}{l}{Objectif Ciblé} &
    Test des nouvelles opérations
    \\
    \midrule
    {\scriptsize \# Test}      &
    \bfseries Entrée           &
    \bfseries Résultat Attendu &
    \bfseries Résultat Obtenu  \\
    \midrule
    \endfirsthead

    \multicolumn{4}{c}%
    {{\itshape \tablename\ \thetable{} -- Continué de la page précédente\ldots}}
    \\
    \midrule
    {\scriptsize \# Test}      &
    \bfseries Entrée           &
    \bfseries Résultat Attendu &
    \bfseries Résultat Obtenu  \\
    \midrule
    \endhead

    \midrule \multicolumn{4}{r}{{Continué à la prochaine page}} \\
    \midrule
    \endfoot

    \bottomrule
    \endlastfoot
    % }}}
    % Tests {{{
    \tidlabel{tid:first} &
    \verb|( U_x - V_y ) * W_z / 35| &
    Sans les parenthèses, l'opérateur de soustraction serait au sommet de
    l'arbre (voir figure \ref{fig:ast-err}). Ce test montre que la priorité des
    opérations est respectée et que les parenthèses modifient bien la
    structure pour obtenir ce qui est observé à la figure \ref{fig:ast-first}. &
    Représentation textuelle du JSON illustré à la figure \ref{fig:ast-first}.
    Elle confirme nos attentes. \\

    \tidlabel{tid:sum}  &
    \verb|1+2-3| &
    Comme la grammaire est associative à droite pour l'addition et la
    soustraction, l'opération au sommet de l'arbre doit être la soustraction,
    placée à droite. &
    Représentation textuelle du JSON illustré à la figure \ref{fig:ast-sum}.
    Elle confirme nos attentes. \\

    \tidlabel{tid:prod} &
    \verb|1*2/3| &
    La grammaire étant associative à droite pour la multiplication et la
    division, l'opération au sommet de l'arbre doit être la division, située
    à droite. &
    Représentation textuelle du JSON illustré à la figure \ref{fig:ast-sum}.
    Elle confirme nos attentes. \\
    % }}}
  \end{longtable}
\end{center}
