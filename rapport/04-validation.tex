\section{Validation et tests}


\begin{center} %% ANALYSE LEXICALE
  \footnotesize
	\begin{longtable}{cp{4cm}p{4cm}p{5cm}}
		% Headers & Footers {{{
		\caption{Tests d'analyse lexicale} \label{tab:lex}
		\\

		\toprule
		\multicolumn{3}{l}{Objectif Ciblé} &
		Test des nouvelles opérations
		\\

		\midrule
    {\scriptsize \# Test}      &
		\bfseries Entrée           &
		\bfseries Résultat Attendu &
		\bfseries Résultat Obtenu  \\

		\midrule
		\endfirsthead

		\multicolumn{4}{c}%
		{{\itshape \tablename\ \thetable{} -- Continué de la page précédente\ldots}}
		\\

		\midrule
    {\scriptsize \# Test}      &
		\bfseries Entrée           &
		\bfseries Résultat Attendu &
		\bfseries Résultat Obtenu  \\

		\midrule
		\endhead

		\midrule \multicolumn{4}{r}{{Continué à la prochaine page}}
		\\
		\midrule
		\endfoot

		\bottomrule
		\endlastfoot
		% }}}
		% Tests {{{
		\tid  &
    \verb|(U_x + V_y ) * W_z / 35| &
    Les unités sont présenté sur une ligne qui contient le type et le
    contenue du lexème le résultat souhaité ici est:
    \begin{itemize}
      \item Délimiteur:     \verb|(|
      \item Identificateur: \verb|U_x|
      \item Opérateur:      \verb|+|
      \item Identificateur: \verb|V_y|
      \item Délimiteur:     \verb|)|
      \item Opérateur:      \verb|*|
      \item Identificateur: \verb|W_z|
      \item Opérateur:      \verb|/|
      \item Littéral:       \verb|35|
    \end{itemize} &
    Le program répond à se critère avec une écriture un peu différente.
    Cependant les données sont présentes et dans le bonne ordre. \\

		\tid  &
    \verb|(U_x + V_y ) * W__z / 35| &
    Les mêmes unités que le premier test. Cependant une erreur est données
    qui indique que des underscores consécutifs ont été trouvés. Les unités
    affichés arrêtent à l’opérateur de multiplication. &
    Le program à donnée une erreur et à retourner les unités lexical jusqu’à
    l’opérateur de multiplication. \\

		\tid  &
    \verb|&| &
    Une erreur de charactère invalide est donnée &
    Le programme identifie le caractère invalide. \\

		\tid  &
    \verb|aA_z| &
    Une erreur est donnée qui indique qu’un identificateur ne peut pas
    commencer par une minuscule. &
    Une erreur d’identificateur est donnée qui dit qu’un identificateur ne
    peut pas commencer par une lettre minuscule. \\

		\tid  &
    \verb|Aa_| &
    Une erreur est donnée qui dit qu’un identificateur ne peut pas terminé
    par un underscore &
    Une erreur de fin d’identificateur à été donnée qui dit que ça peut pas
    terminé par un underscore.  \\
    % }}}
	\end{longtable}
\end{center}

\begin{center} %% ANALYSE SYNTAXIQUE
  \footnotesize
	\begin{longtable}{cp{4cm}p{4cm}p{5cm}}
		% Headers & Footers {{{
		\caption{Tests d'analyse syntaxique} \label{tab:syntax}
		\\

		\toprule
		\multicolumn{3}{l}{Objectif Ciblé} &
		Test des nouvelles opérations
		\\

		\midrule
    {\scriptsize \# Test}      &
		\bfseries Entrée           &
		\bfseries Résultat Attendu &
		\bfseries Résultat Obtenu  \\

		\midrule
		\endfirsthead

		\multicolumn{4}{c}%
		{{\itshape \tablename\ \thetable{} -- Continué de la page précédente\ldots}}
		\\

		\midrule
    {\scriptsize \# Test}      &
		\bfseries Entrée           &
		\bfseries Résultat Attendu &
		\bfseries Résultat Obtenu  \\

		\midrule
		\endhead

		\midrule \multicolumn{4}{r}{{Continué à la prochaine page}}
		\\
		\midrule
		\endfoot

		\bottomrule
		\endlastfoot
		% }}}
		% Tests {{{
    \tidlabel{tid:start-op} &
    \verb| + 2 - 3| &
    Une expression ne peut pas commencer par un opérateur. Une erreur devrait survenir. &
    Voir l'erreur \ref{err:start-op} \\

    \tidlabel{tid:end-op}  &
    \verb| 2 - 3 *| &
    Un opérateur est utilisé sans opérande droit. Une erreur devrait survenir. &
    Voir l'erreur \ref{err:end-op} \\

    \tidlabel{tid:double-op}  &
    \verb|1 +/ 2 - 3| &
    Deux opérateurs consécutifs sont invalides. Une erreur devrait survenir. &
    Voir l'erreur \ref{err:double-op} \\

    \tidlabel{tid:missing-close}  &
    \verb|(1 + 2 - 3| &
    Un parenthèse ouverte n'a jamais été fermée. Une erreur devrait survenir. &
    Voir l'erreur \ref{err:missing-close} \\


    \tidlabel{tid:missing-open}  &
    \verb|1 + 2) - 3| &
    Un parenthèse ferme un alors qu'aucune parenthèse n'a été ouverte. Une
    erreur devrait survenir. &
    Voir l'erreur \ref{err:missing-open} \\

    \tidlabel{tid:empty-paren}  &
    \verb|1 + 2 - () / 3| &
    Des parenthèses doivent contenir une expression valide. Une erreur
    devrait survenir. &
    Voir l'erreur \ref{err:empty-paren} \\

    % }}}
	\end{longtable}
\end{center}

Des test syntaxique avec une syntaxe valide n'ont pas été effectués puisque
tout les test de construction d'AST les emploient (un prérequis pour ce type
de test).

\begin{center} %% CONSTRUCTION AST
  \footnotesize
	\begin{longtable}{cp{4cm}p{4cm}p{5cm}}
		% Headers & Footers {{{
		\caption{Tests de construction d'arbres syntaxiques abstraits} \label{tab:ast}
		\\

		\toprule
		\multicolumn{3}{l}{Objectif Ciblé} &
		Test des nouvelles opérations
		\\

		\midrule
    {\scriptsize \# Test}      &
		\bfseries Entrée           &
		\bfseries Résultat Attendu &
		\bfseries Résultat Obtenu  \\

		\midrule
		\endfirsthead

		\multicolumn{4}{c}%
		{{\itshape \tablename\ \thetable{} -- Continué de la page précédente\ldots}}
		\\

		\midrule
    {\scriptsize \# Test}      &
		\bfseries Entrée           &
		\bfseries Résultat Attendu &
		\bfseries Résultat Obtenu  \\

		\midrule
		\endhead

		\midrule \multicolumn{4}{r}{{Continué à la prochaine page}}
		\\
		\midrule
		\endfoot

		\bottomrule
		\endlastfoot
		% }}}
		% Tests {{{
    \tidlabel{tid:first} &
    \verb|( U_x - V_y ) * W_z / 35| &
    Si les parenthèse n'étaient pas présente, l'opérateur de soustraction se
    retrouverait au sommet de l'arbre comme à la figure \ref{fig:ast-err}.
    Ce test permet d'affirmer que la priorité d'opération est respectée et que
    les parenthèes changent la structure de la pour obtenir ce qui est observé
    à la figure \ref{fig:ast-first}. &
    Représentation textuelle du json représenté à la figure \ref{fig:ast-first}.
    Celle-ci confirme nos attentes \\

    \tidlabel{tid:sum}  &
    \verb|1+2-3| &
    Puisque la grammaire définie est associative à droite pour l'addition et
    la soustraction, le terme au sommet de l'arbre devrait être la soustraction
    qui se situe à droite. &
    Représentation textuelle du json représenté à la figure \ref{fig:ast-sum}.
    Celle-ci confirme nos attentes \\

    \tidlabel{tid:prod} &
    \verb|1*2/3| &
    Puisque la grammaire définie est associative à droite pour la multiplication
    et la division, le terme au sommet de l'arbre devrait être la division qui
    se situe à droite. &
    Représentation textuelle du json représenté à la figure \ref{fig:ast-sum}.
    Celle-ci confirme nos attentes \\

    % }}}
	\end{longtable}
\end{center}

